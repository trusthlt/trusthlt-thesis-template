% !TeX program = pdflatex
% !BIB program = biber
\documentclass[a4paper,11pt,oneside]{book}

% make sure we use utf-8
\usepackage[utf8]{inputenc}

% for german declaraion as second language
\usepackage[main=english,ngerman]{babel}

% figures
\usepackage{graphicx}

% lorem ipsum
\usepackage{lipsum}

% better bibliography using biber as backend
\usepackage[natbib=true,backend=biber,style=authoryear-icomp,maxbibnames=9,maxcitenames=2,uniquelist=false,uniquename=init,
giveninits=true,doi=true,url=true,dashed=false,isbn=false]{biblatex}
\addbibresource{bibliography.bib}
% disable "ibid" for repeated citations
\boolfalse{citetracker}

% nice tables
\usepackage{booktabs}

% shorten
\usepackage{microtype}

% for compact item
\usepackage{paralist}

% palatino font
\usepackage[sc]{mathpazo}

% for \theauthor \thetitle etc.
\usepackage{titling}

% for colors
\usepackage{color}

\author{Alice Maria Smith-Doe von~Großburg de~Montéz}
\title{Statistical Evaluation of Coffee Intake as a Predictor of Academic Productivity}
\date{January 1, 2026}

\begin{document}

\begin{titlepage}
\centering
\vspace*{\fill}

{\fontsize{32}{40}\selectfont
\begin{minipage}{0.95\textwidth}
\centering
\thetitle
\end{minipage}
}

\vspace{6em}

{\fontsize{16}{20}\selectfont
% Master's thesis by \theauthor
Bachelor's thesis by \theauthor
}


\vspace{4em}

Immatriculation number 1234567891234 \\
Submitted on \thedate


\vspace{4em}

First reviewer: Prof.\ Dr.\ Ivan Habernal

Second reviewer: John Doe

\vspace{4em}

Trustworthy Human Language Technologies

Faculty of Computer Science

Ruhr University Bochum

\vspace{4em}


\includegraphics[width=5cm]{img/rub-logo.pdf}


\vspace*{\fill}
\end{titlepage}


\begin{titlepage}

% just for this page
{
\setlength{\parindent}{0em}
\setlength{\parskip}{0em}




\section*{Eidesstattliche Erklärung---Statutory Declaration}

\subsection*{Titel meiner Abschlussarbeit---Title of the final thesis}

\thetitle

\begin{otherlanguage}{ngerman}
\subsection*{Eidesstattliche Erklärung}

Ich erkläre, dass ich keine Arbeit in gleicher oder ähnlicher Fassung bereits für eine andere Prüfung an der Ruhr-
Universität Bochum oder einer anderen Hochschule zur Erlangung eines akademischen Grades eingereicht habe.

\medskip

Ich versichere, dass ich diese Arbeit selbstständig verfasst und keine anderen als die angegebenen Quellen benutzt
habe. Die Stellen, die anderen Quellen dem Wortlaut oder dem Sinn nach entnommen sind, habe ich unter Angabe der
Quellen kenntlich gemacht. Dies gilt sinngemäß auch für verwendete Zeichnungen, Skizzen, bildliche Darstellungen
und dergleichen.

\medskip

Ich erkläre mich des Weiteren damit einverstanden, dass die digitale Version dieser Arbeit zwecks Plagiatsprüfung
verwendet wird.
\end{otherlanguage}

\subsection*{Statutory Declaration}

Hereby I declare, that I have not submitted this thesis in this or similar form to any other examination at the Ruhr-
Universität Bochum or any other institution or university to obtain an academic degree.

\medskip

I officially ensure, that this paper has been written solely on my own. I herewith officially ensure, that I have not
used any other sources but those stated by me. Any and every parts of the text which constitute quotes in original
wording or in its essence have been explicitly referred by me by using official marking and proper quotation. This is
also valid for used drafts, pictures and similar formats.

\medskip

I furthermore agree that the digital version of this thesis will be used to subject the paper to plagiarism examination.

\medskip

Not this English translation but only the official version in German is legally binding.

\bigskip

\begin{tabular}{ll}
Date: & \thedate \\
Name: &\theauthor \vspace*{1em} \\
Signature: & \\
\end{tabular}


} % enf for this page

\end{titlepage}


\frontmatter

\section*{Abstract}

\lipsum[1]

\tableofcontents

\mainmatter

\chapter{The First Chapter}

\lipsum[1]

Paragraphs are separated by at least one empty line, do not use backslashes for that.

\section{Some section}

Citing passively, just at the end of the sentence \citep{Habernal.Gurevych.2017.COLI,Habernal.Gurevych.2016.ACL}. But sometimes you want say something, as suggested by \citet{Habernal.Gurevych.2016.EMNLP}.

\subsection{Some subsection}

\lipsum[1]

\begin{itemize}
\item Item 1
\item Item 2
\end{itemize}

\lipsum[1]

\subsubsection{Some sub-subsection}

\lipsum[1]

\subsubsection{Another sub-subsection}

\lipsum[1]

\paragraph{And even paragraph level}

\lipsum[1]

\chapter{The Second Chapter}

Let's talk about tables. Each table must be referenced, just like Table~\ref{tab.1}.

\begin{table}
\centering
\begin{tabular}{lrr}
\toprule
Method & Accuracy & Other metric \\
\midrule
GPT-5 & 99.2 & 12.6 \\
GPT-10 & 99.8 & 16.7 \\
\midrule
\end{tabular}
\caption{\label{tab.1} Some table. No vertical lines. Make sure numbers are right-aligned, texts are left-aligned. Don't position the table by \texttt{[h]} or alike, just let \LaTeX\ to do the job for you.}
\end{table}


\chapter{Example of an entire paper}

I adopted the \LaTeX\ source codes and figures from my older NAACL 2018 paper ``Before Name-Calling: Dynamics and Triggers of Ad Hominem Fallacies in Web Argumentation'' (\url{https://aclanthology.org/N18-1036/}).


\section{Introduction}
\label{sec:introduction}

Human reasoning is lazy and biased but it perfectly serves its purpose in the argumentative context \citep{Mercier.Sperber.2017.book}. When challenged by genuine back-and-forth argumentation, humans do better in both generating and evaluating arguments \citep{Mercier.2011}. The dialogical perspective on argumentation has been reflected in argumentation theory prominently by the pragma-dialectic model of argumentation \citep{vanEemeren.Grootendorst.1992}. Not only sketches this theory an ideal normative model of argumentation but also distinguishes the wrong argumentative moves, \emph{fallacies} \citep{vanEemeren.Grootendorst.1987}. Among the plethora of prototypical fallacies, notwithstanding the controversy of most taxonomies \citep{Boudry.et.al.2015}, \emph{ad hominem} argument is perhaps the most famous one. Arguing \emph{against the person} is considered faulty, yet is prevalent in online and offline discourse.\footnote{According to `Godwin's law' known from the internet pop-culture (\url{https://en.wikipedia.org/wiki/Godwin's_law}), if a discussion goes on long enough, sooner or later someone will compare someone or something to Adolf Hitler.}


Although the ad hominem fallacy has been known since Aristotle, surprisingly there are very few empirical works investigating its properties. While \citet{Sahlane.2012} analyzed ad hominem and other fallacies in several hundred newspaper editorials, others usually only rely on few examples, as observed by \citet{DeWijze.2002}. As \citet{Macagno.2013.ArgJournal} concludes, ad hominem arguments should be considered as multifaceted and complex strategies, involving not a simple argument, but several
combined tactics. However, such research, to the best of our knowledge, does not exist. Very little is known not only about the feasibility of ad hominem theories in practical applications (the NLP perspective) but also about the dynamics and triggers of ad hominem (the theoretical counterpart).


This paper investigates the research gap at three levels of increasing discourse complexity: ad hominem in isolation, direct ad hominem without dialogical exchange, and ad hominem in large inter-personal discourse context. We asked the following research questions. First, what qualitative and quantative properties do ad hominem arguments have in Web debates and how does that reflect the common theoretical view (RQ1)? Second, how much of the debate context do we need for recognizing ad hominem by humans and machine learning systems (RQ2)? And finally, what are the actual triggers of ad hominem arguments and can we predict whether the discussion is going to end up with one (RQ3)?

We tackle these questions by leveraging Web-based argumentation data (\emph{Change my View} on Reddit), performing several large-scale annotation studies, and creating a new dataset. We experiment with various neural architectures and extrapolate the trained models to validate our working hypotheses. Furthermore, we propose a list of potential linguistic and rhetorical triggers of ad hominem based on interpreting parameters of trained neural models.\footnote{An attempt to address the plea for thinking about problems, cognitive science, and the details of human language \citep{Manning.2015.CoLi}.} This article thus presents the first NLP work on multi-faceted ad hominem fallacies in genuine dialogical argumentation. We also release the data and the source code to the research community.\footnote{\url{https://github.com/UKPLab/naacl2018-before-name-calling-habernal-et-al}}


\section{Theoretical background and related work}
\label{sec:related.work}

The prevalent view on argumentation emphasizes its pragmatic goals, such as persuasion and group-based deliberation \citep{vanEemeren.et.al.2014}, although numerous works have dealt with argument as product, that is, treating a single argument and its properties in isolation \citep{Toulmin.1958,Habernal.Gurevych.2017.COLI}. Yet the social role of argumentation and its alleged responsibility for the very skill of human reasoning explained from the evolutionary perspective \citep{Mercier.Sperber.2017.book} provide convincing reasons to treat argumentation as an inherently dialogical tool.

The observation that some arguments are in fact `deceptions in disguise' was made already by Aristotle \citep{Aristotle.1991}, for which the term \emph{fallacy} has been adopted. Leaving the controversial typology of fallacies aside \citep{Hamblin.1970,vanEemeren.Grootendorst.1987,Boudry.et.al.2015}, the \emph{ad hominem} argument is addressed in most theories.
Ad hominem argumentation relies on the strategy of attacking the opponent and some feature of the opponent's character instead of the counter-arguments \citep{Tindale.2007}. With few exceptions, the following five sub-types of ad hominem are prevalent in the literature:
\textbf{abusive ad hominem} (a pure attack on the character of the opponent), 
\textbf{tu quoque ad hominem} (essentially analogous to the ``He did it first'' defense of a three-year-old in a sandbox), \textbf{circumstantial ad hominem} (the ``practice what you preach'' attack and accusation of hypocrisy),
\textbf{bias ad hominem} (the attacked opponent has a hidden agenda), and
\textbf{guilt by association} (associating the opponent with somebody with a low credibility) \citep{Schiappa.Nordin.2013,Macagno.2013.ArgJournal,Walton.2007a,Hansen.2017,Woods.2008}. We omit examples here as these provided in theoretical works or textbooks are usually artificial, as already criticized by \citep{DeWijze.2002} or \citep{Boudry.et.al.2015}.

The topic of fallacies, which might be considered as sub-topic of argumentation quality, has recently been investigated also in the NLP field. Existing works are, however, limited to the monological view \citep{Wachsmuth.et.al.2017.ACL,Habernal.Gurevych.2016.ACL,Habernal.Gurevych.2016.EMNLP,Stab.Gurevych.2017.EACL} or they focus primarily on learning fallacy recognition by humans \citep{Habernal.et.al.2017.EMNLP,Habernal.et.al.2018.LREC}.
Another related NLP sub-field includes abusive language and personal attacks in general.
\citet{Wulczyn.et.al.2017.WWW} investigated whether or not Wikipedia talk page comments are personal attacks and annotated 38k instances resulting in a highly skewed distribution (only 0.9\% were actual attacks). Regarding the participants' perspective, \citet{Jain.et.al.2014.LREC} examined principal roles in 80 discussions from the \emph{Wikipedia: Article for Deletion} pages (focusing on stubbornness or ignoredness, among others) and found several typical roles, including `rebels', `voices', or `idiots'. In contrast to our data under investigation (Change My View debates), Wikipedia talk pages do not adhere to strict argumentation rules with manual moderation and have a different pragmatic purpose.

Reddit as a source platform has also been used in other relevant works. \citet{Saleem.et.al.2016.LREC.WS} detected hateful speech on Reddit by exploiting particular sub-communities to automatically obtain training data.
\citet{Wang.et.al.2016.NLP.SocSci.WS} experimented with an unsupervised neural model to cluster social roles on sub-reddits dedicated to computer games.
\citet{Zhang.et.al.2017.ICWSM} proposed a set of nine comment-level dialogue act categories and annotated 9k threads with 100k comments and built a CRF classifier for dialogue act labeling.
Unlike these works which were not related to argumentation, \citet{Tan.2016} examined persuasion strategies on Change My View using word overlap features.
In contrast to our work, they focused solely on the successful strategies with delta-awarded posts.
Using the same dataset, \citet{Musi.2017} recently studied concession in argumentation.


\section{Data}
\label{sec:data}

\emph{Change My View} (CMV) is an online `place to post an opinion you accept [...] in an effort to understand other perspectives on the issue', in other words an online platform for `good-faith' argumentation hosted on Reddit.\footnote{\url{https://www.reddit.com/r/changemyview/}} A user posts a \textbf{submission} (also called \textbf{original post(er); OP}) and other participants provide arguments to change the OP's view, forming a typical tree-form Web discussion. A special feature of CMV is that the OP acknowledges convincing arguments by giving a \textbf{delta} point ($\Delta$). Unlike the vast majority of internet discussion forums, CMV enforces obeying strict rules (such as no `low effort' posts, or accusing of being unwilling to change view) whose violation results into deleting the comment by moderators.
These formal requirements of an ideal debate with the notion of violating rules correspond to incorrect moves in critical discussion in the normative pragma-dialectic theory \citep{vanEemeren.Grootendorst.1987}. \emph{Thus, violating the rule of `not being rude or hostile' is equivalent to committing ad hominem fallacy.}
For our experiments, we scraped, in cooperation with Reddit, the complete CMV including the content of the deleted comments so we could fully reconstruct the fallacious discussions, relying on the rule violation labels provided by the moderators. The dataset contains $\approx$ 2M posts in 32k submissions, forming 780k unique threads.



We will set up the stage for further experiments by providing several quantitative statistics we performed on the dataset. Only 0.2\% posts in CMV are ad hominem arguments. This contrasts with a typical online discussion: \citet{Coe.et.al.2014} found 19.5\% of comments under online news articles to be incivil. Most threads contain only a single ad hominem argument (3,396 threads; there are 3,866 ad hominem arguments in total in CMV); only 35 threads contain more than three ad hominem arguments. In 48.6\% of threads containing a single ad hominem, the ad hominem argument is the very last comment. This corresponds to the popular belief that if one is out of arguments, they start attacking and the discussion is over. This trend is also shown in Figure \ref{fig:before-ah-thread} which displays the relative position of the first ad hominem argument in a thread. Replying to ad hominem with another ad hominem happens only in 15\% of the cases; this speaks for the attempts of CMV participants to keep up with the standards of a rather rational discussion.


\begin{figure}
\centering
\includegraphics[width=0.5\textwidth]{img/relative-position-ah-thread}
\caption{\label{fig:before-ah-thread} `No discussion after ad hominem.' Distribution of the number of comments before the first ad hominem is committed proportional to the thread length.}
\end{figure}

Regarding ad hominem authors, about 66\% of them start attacking `out of blue', without any previous interaction in the thread. On the other hand, 11\% ad hominem authors write at least one `normal' argument in the thread (we found one outlier who committed ad hominem after writing 57 normal arguments in the thread).
Only in 20\% cases, the ad hominem thread is an interplay between the original poster and another participant. It means that there are usually more people involved in an ad hominem thread. Unfortunately, sometimes the OP herself also commits ad hominem (12\%).


We also investigated the relation between the presence of ad hominem arguments and the submission topic. While most submissions are accompanied by only one or two ad hominem arguments (75\% of submissions), there are also extremes with over 50 ad hominem arguments. Manual analysis revealed that these extremes deal with religion, sexuality/gender, U.S. politics (mostly Trump), racism in the U.S., and veganism. We will elaborate on that later in Section \ref{sec:triggers.first.level}.


\section{Experiments}

The experimental part is divided into three parts according to the increasing level of discourse complexity. We first experiment with ad hominem in isolation in section \ref{sec:ad.hominem.in.cmv}, then with direct ad hominem replies to original posts without dialogical exchange in section \ref{sec:triggers.first.level}, and finally with ad hominem in a larger inter-personal discourse context in section \ref{sec:before.calling.names}.


\subsection{Ad hominem without context in CMV}
\label{sec:ad.hominem.in.cmv}

The first experimental set-up examines ad hominem arguments in \emph{Change my view} regardless of its dialogical context.

\subsubsection{Data verification}
\label{sec:data.verification}

Ad hominem arguments labeled by the CMV moderators come with no warranty. To verify their reliability, we conducted the following annotation studies. First, we needed to estimate parameters of crowdsourcing and its reliability. We sampled 100 random arguments from CMV without context: positive candidates were the reported ad hominem arguments, whereas negative candidates were sampled from comments that either violate other argumentation rules or have a delta label. To ensure the maximal content similarity of these two groups, for each positive instance the semantically closest negative instance was selected.\footnote{Similarity was computed using a cosine similarity of average embedding vectors multiplied by the argument length difference to minimize length-related artifacts. The sample was balanced with roughly 50\% positive and 50\% negative instances.} We then experimented with different numbers of Amazon Mechanical Turk workers and various thresholds of the MACE gold label estimator \citep{Hovy.et.al.2013}; comparing two groups of six workers each and 0.9 threshold yielded almost perfect inter-annotator agreement (0.79 Cohen's $\kappa$). We then used this setting (six workers, 0.9 MACE threshold) to annotate another 452 random arguments sampled in the same way as above.

Crowdsourced `gold' labels were then compared to the original CMV labels (balanced binary task: positive instances (ad hominem) and negative instances) reaching accuracy of 0.878. This means that the ad hominem labels from CMV moderators are quite reliable. Manual error analysis of disagreements revealed 11 missing ad hominem labels. These were not spotted by the moderators but were annotated as such by crowd workers.




\subsubsection{Recognizing ad hominem arguments}
\label{sec:recognizing.ah}

We sampled a larger balanced set of positive instances (ad hominem) and negative instances using the same methodology as in section \ref{sec:data.verification}, resulting in 7,242 instances, and casted the task of recognition of ad hominem arguments as a binary supervised task. We trained two neural classifiers, namely a 2-stacked bi-directional LSTM network \citep{Graves.Schmidthuber.2005}, and a convolutional network \citep{Kim.2014}, and evaluated them using 10-fold cross validation. Throughout the paper we use pre-trained \texttt{word2vec} word embeddings \citep{Mikolov.2013}. Detailed hyperparameters are described in the source codes (link provided in section \ref{sec:introduction}). As results in Table \ref{tab:ah-prediction} show, the task of recognizing ad hominem arguments is feasible and almost achieves the human upper bound performance.

\begin{table}
\centering
\begin{small}
\begin{tabular}{lr}
\toprule
\textbf{Model} & \textbf{Accuracy} \\
\midrule
Human upper bound estimate & 0.878 \\
2 Stacked Bi-LSTM & 0.782 \\
CNN & \textbf{0.810} \\
\bottomrule
\end{tabular}
\end{small}
\caption{\label{tab:ah-prediction} Prediction of ad hominem arguments}
\end{table}



\subsubsection{Typology of ad hominem}
\label{sec:typology.of.ah}

While binary classification of ad hominem as presented above might be sufficient for the purpose of red-flagging arguments, theories provide us with a much finer granularity (recall the typology in section \ref{sec:related.work}). To validate whether this typology is empirically relevant, we executed an annotation experiment to classify ad hominem arguments into the provided five types (plus `other' if none applies). We sampled 200 ad hominem arguments from threads in which interlocution happens only between two persons and which end up with ad hominem. The Mechanical Turk workers were shown this last ad hominem argument as well as the preceding one. Each instance was annotated by 16 workers to achieve a stable distribution of labels as suggested by \citet{Aroyo.Welty.2015}. While 41\% arguments were categorized as \emph{abusive}, other categories (\emph{tu quoque}, \emph{circumstantial}, and \emph{guilt by association}) were found to be rather ambiguous with very subtle differences. In particular, we observed a very low percentage agreement on these categories and a label distribution spiked around two or more categories. After a manual inspection we concluded that (1) the theoretical typology does not account for longer ad hominem arguments that mix up different attacks and that (2) there are actual phenomena in ad hominem arguments not covered by theoretical categories. These observations reflect those of \citet[p.~399]{Macagno.2013.ArgJournal} about ad hominem moves as multifaceted strategies.


We thus propose a list of phenomena typical to ad hominem arguments in CMV based on our empirical study. For this purpose, we follow up with another annotation experiment on 400 arguments, with seven workers per instance.\footnote{Here we decided on seven workers per item by relying on other span annotation experiments done in a similar setup \citep{Habernal.et.al.2018.NAACL.Reasoning}.} The goal was to annotate a text span which made the argument an ad hominem; a single argument could contain several spans. We estimated the gold spans using MACE and performed a manual post-analysis by designing a typology of causes of ad hominem together with their frequency of occurrence. The results and examples are summarized in Table \ref{tab:typology-spans}.



\begin{table}
\centering
\begin{footnotesize}
\begin{tabular}{p{11em}rp{23em}}
\toprule
\textbf{Type} &\textbf{ (\%)} & \textbf{Example spans} \\ \midrule
Vulgar insult & 31.3 & "Your just an asshole", "you dumb fuck", etc. \\ \midrule
Illiteracy insult & 13.0 & "Reading comprehension is your friend", "If you can't grasp the concept, I can't help you" \\ \midrule
Condescension & 6.5 & "little buddy", "sir", "boy", "Again, how old are you?" \\ \midrule
Ridiculing and sarcasm & 6.5 & "Thank you so much for all your pretentious explanations", "Can you also use Google?" \\ \midrule
`Idiot'-insults & 6.5 & "Ever have discussions with narcissistic idiots on the internet? They are so tiring" \\ \midrule
Accusation of stupidity & 4.3 & "You have no capability to understand why", "You're obviously just Nobody with enough brains to operate a computer could possibly believe something this stupid" \\ \midrule
Lack of argumentation skills & 4.3 & "You're making the claims, it's your job to prove it. Don't you know how debating works?", "You're trash at debating." \\ \midrule
Accusation of trolling & 3.9 & "You're just a dishonest troll", "You're using troll tactics" \\ \midrule
Accusation of ignorance & 3.5 & "Please dont waste peoples time pretending to know what you're talking about", "Do you even know what you're saying?" \\ \midrule
"You didn't read what I wrote" & 3.0 & "Read what I posted before acting like a pompous ass", "Did you even read this?" \\ \midrule
"What you say is idiotic" & 2.6 & "To say that people intrinsically understand portion size is idiotic.", "Your second paragraph is fairly idiotic" \\ \midrule
Accusation of lying & 2.6 & "Possible lie any harder?", "You are just a liar." \\ \midrule
"You don't face the facts and ignore the obvious" & 1.7 & "Willful ignorance is not something I can combat", "How can you explain that? You can't because it will hurt your feelings to face reality" \\ \midrule
Accusation of ad hominem or other fallacies & 1.7 & "You started with a fallacy and then deflected.", "You still refuse to acknowledge that you used a strawman argument against me" \\ \midrule
Other & 8.3 & "Wow. Someone sounds like a bit of an anti-semite", "You're too dishonest to actually quote the verse because you know it's bullshit" \\
\bottomrule
\end{tabular}
\end{footnotesize}
\caption{\label{tab:typology-spans} What makes an argument ad hominem: results of the empirical study of labeling spans in 400 ad hominem arguments.}
\end{table}


\subsubsection{Results and interpretation}

The data verification annotation study (section \ref{sec:data.verification}) has two direct consequences. First, the high $\kappa$ score (0.79) answers RQ2: for recognizing ad hominem argument, no previous context is necessary. Second, we still found 5\% overlooked ad hominem arguments in CMV thus a moderation-facilitating tool might come handy; this can be served by the well-performing CNN model (0.810 accuracy; section \ref{sec:recognizing.ah}).

The existing theoretical typology of ad hominem arguments, as presented for example in most textbooks, provides only a very simplified view. On the one hand, some of the categories which we found in the empirical labeling study (section \ref{sec:typology.of.ah}) do map to their corresponding counterparts (such as the vulgar insults). On the other hand, some ad hominem insults typical to online argumentation (illiteracy insults, condescension) are not present in studies on ad hominem. Hence, we claim that any potential typology of ad hominem arguments should be multinomial rather than categorical, as we found multiple different spans in a single argument.


\subsection{Triggers of first level ad hominem}
\label{sec:triggers.first.level}

In the following section, we increase the complexity of the studied discourse by taking the original post into account.

\subsubsection{Annotation study}

We already showed that ad hominem arguments are usually preceded by a discussion between the interlocutors. However, 897 submissions (original posts; OPs) have at least one intermediate ad hominem (in other words, the original post is directly attacked). We were thus interested in what triggers these first-level ad hominem arguments. We hypothesize two causes: (1) the \emph{controversy} of the OP, similarly to some related works on news comments \citep{Coe.et.al.2014} and (2) the \emph{reasonableness} of the OP (whether the topic is reasonable to argue about). We model both features on a three-point scale, namely \emph{controversy}: 1 = `not really controversial', 2 = `somehow controversial', 3 = `very controversial' and \emph{reasonableness}: 1 = `quite stupid', 2 = `neutral', 3 = `quite reasonable'.\footnote{Examples of not really controversial: \emph{"I Don't Think Monty Python is Funny"}, very controversial: \emph{"Blacks are generally intellectual inferior to the other major races"}, quite stupid: \emph{"Burritos are better than sandwiches"}, and quite reasonable: \emph{"Nations whose leadership is based upon religion are fundamentally backwards"}.}

We sampled two groups of OPs: those which had some ad hominem arguments in any of its threads but no delta (ad hominem group) and those without ad hominem but some deltas (Delta group). In total, 1,800 balanced instances were annotated by five workers and the resulting value was averaged for each item.\footnote{A pilot crowd sourcing annotation with 5 + 5 workers showed a fair reliability for controversy (Spearman's $\rho$ 0.804) and medium reliability for reasonableness (Spearman's $\rho$ 0.646).}


Statistical analysis of the annotated 1,800 OPs revealed that ad hominem arguments are associated with more controversial OPs (mean controversy 1.23) while delta-awarded arguments with less controversial OPs (mean controversy 1.06; K-S test;\footnote{Kolmogorov-Smirnov (K-S) test is a non-parametric test without any assumptions about the underlying probability distribution.} statistics 0.13, P-value: $7.97\times10^{-7}$).
On the other hand, reasonableness does not seem to play such a role. The difference between ad hominem in reasonable OPs (mean 1.20) and delta in reasonable OPs (mean 1.11) is not that statistically strong; (K-S test statistics: 0.07, P-value: 0.02).


\subsubsection{Regression model for predicting controversy and reasonableness}

We further built a regression model for predicting controversy and reasonableness of the OPs. Along with Bi-LSTM and CNN networks (same models as in \ref{sec:recognizing.ah}) we also developed a neural model that integrates CNN with topic distribution (CNN+LDA). The motivation for a topic-incorporating model was based on our earlier observations presented in section \ref{sec:data}. In particular, we trained an LDA topic model ($k$ = 50) \citep{Blei.et.al.2003} on the heldout OPs and during training/testing, we merged the estimated topic distribution vector with the output layer after convolution and pooling. We performed 10-fold cross validation on the 1,800 annotated OPs and got reasonable performance for controversy prediction ($\rho$ 0.569) and medium performance for reasonableness prediction ($\rho$ 0.385), respectively; both using the CNN+LDA model (see Table \ref{tab:controversy-reasonableness}).

\begin{table}
\centering
\begin{small}
\begin{tabular}{lr}
\toprule
\multicolumn{2}{c}{\textbf{Controversy (Spearman's $\rho$)}} \\ \midrule
Human upper bounds & 0.804 \\
Bi-LSTM & 0.539 \\
CNN	& 0.559 \\
CNN-LDA	& \textbf{0.569} \\ \midrule
\multicolumn{2}{c}{\textbf{Reasonableness (Spearman's $\rho$)}}  \\ \midrule
Human upper bounds & 0.646 \\
Bi-LSTM	& 0.332 \\
CNN	& 0.320 \\
CNN-LDA & \textbf{0.385} \\ \bottomrule
\end{tabular}
\end{small}
\caption{\label{tab:controversy-reasonableness} Results of predicting controversy and reasonableness of the original post.}
\end{table}


We then used the trained model and extrapolated on all held-out OPs (1,267 ad hominem and 10,861 delta OPs, respectively). The analysis again showed that ad hominem arguments tend to be found under more controversial OPs whereas delta arguments in the less controversial ones (K-S test statistics: 0.14, P-value: $1\times10^{-18}$). For reasonableness, the rather low performance of the predictor does not allow us draw any conclusions on the extrapolated data.


\subsubsection{Results and interpretation}

Controversy of the original post is immediately heating up the debate participants and correlates with a higher number of direct ad hominem responses. This corresponds to observations made in comments in newswire where `weightier' topics tended to stir incivility \citep{Coe.et.al.2014}. On the other hand, `stupidity' (or `reasonableness') does not seem to play any significant role. The CNN+LDA model for predicting controversy ($\rho$ 0.569) might come handy for signaling potentially `heated' discussions.


\subsection{Before calling names}
\label{sec:before.calling.names}

In this section, we focus on the dialogical aspect of CMV debates and dynamics of ad hominem fallacies. Although ad hominem arguments appear in many forms (Section \ref{sec:typology.of.ah}), we treat all ad hominem arguments equal in the following experiments.

\subsubsection{Data sampling}
So far we explored what makes an ad hominem argument and whether debated topic influences the number of intermediate attacks. However, possible causes of the argumentative dynamics that ends up with an ad hominem argument remain an open question, which has been addressed in neither argumentation theory nor in cognitive psychology, to the best of our knowledge. We thus cast an explanation of triggers and dynamics of ad hominem discussions as a supervised machine learning problem and draw theoretical insights by a retrospective interpretation of the learned models.

We sample positive instances by taking three contextual arguments preceding the ad hominem argument from threads which are an interplay between two persons. Negative samples are drawn similarly from threads in which the argument is awarded with $\Delta$ as shown in Figure \ref{fig:thread-sampling}.\footnote{To ensure as much content similarity as possible, we used the same similarity sampling as in section \ref{sec:data.verification}.} Each instance consists of the three concatenated arguments delimited by a special OOV token. This resulted in 2,582 balanced training instances.

\begin{figure}
\centering
\includegraphics[width=0.8\textwidth]{img/sampling-threads}
\caption{\label{fig:thread-sampling} Sampling instances for learning triggers of ad hominem.}
\end{figure}

\subsubsection{Neural models}

The alleged lack of interpretability of neural networks has motivated several lines of approaches, such as layer-wise relevance propagation \citep{Arras.et.al.2017.WASSA} or representation erasure \citep{Li.et.al.2016.arXiv}, both on sentiment analysis.
As our task at hand deals with multi-party discourse that presumably involves temporal relations important for the learned representation, we opted for a state-of-the-art self-attentive LSTM model. In particular, we re-implemented the Structured Self-Attentive Embedding Neural Network (SSAE-NN) \citep{Lin.et.al.2017.ICLR} which learns an embedding matrix representation of the input using attention weights. To make the attention even more interpretable, we replaced the final non-linear MLP layers with a single linear classifier (softmax). By summing over one dimension of the attention embedding matrix, each word from the input sequence gets associated with a single attention weight that gives us insights into the classifier's `features' (still indirectly, as the true representation is a matrix; see the original paper).\footnote{We also experimented with regularizing the attention matrix as the authors proposed, but it resulted in worse performance.} The learning objective is to recognize whether the thread ends up in an ad hominem argument or a delta point. We trained the model in 10-fold cross-validation and although our goal is not to achieve the best performance but rather to gain insight, we also tested a CNN model (accuracy 0.7095) which performed slightly worse than the SSAE-NN model (accuracy 0.7208).


\subsubsection{Results and interpretation}
\label{sec:before.calling.names:results}

%\parsum{What can we draw from it?}
During testing the model, we projected attention weights to the original texts as heat maps and manually analyzed 191 true positives (ad hominem threads recognized correctly), as well as 77 false positives (ad hominem threads misclassified as delta) and 84 false negatives (delta as ad hominem), in total about 120k tokens. The full output is available in the supplementary materials, we use IDs as a reference in the following text.




\begin{figure}

\begin{small}
\textbf{587\_ah\_t1\_cm7djx3}

\noindent(OOV\_comment\_begin) If only you would n't rely on [ \colorbox[rgb]{0.36,0.42,0.95}{\strut \textcolor{white}{fallacious}} \colorbox[rgb]{0.79,0.80,0.95}{\strut ]} ( http : OOV ) [ \colorbox[rgb]{0.82,0.84,0.95}{\strut arguments} ] ( http : OOV ) to make your point. \colorbox[rgb]{0.81,0.82,0.95}{\strut So} \colorbox[rgb]{0.82,0.83,0.95}{\strut no} \colorbox[rgb]{0.77,0.79,0.95}{\strut ,} \colorbox[rgb]{0.53,0.57,0.95}{\strut \textcolor{white}{I}} \colorbox[rgb]{0.81,0.83,0.95}{\strut do} \colorbox[rgb]{0.77,0.79,0.95}{\strut n't} \colorbox[rgb]{0.74,0.76,0.95}{\strut realize} \colorbox[rgb]{0.84,0.85,0.95}{\strut how} \colorbox[rgb]{0.67,0.69,0.95}{\strut stupid} \colorbox[rgb]{0.84,0.85,0.95}{\strut and} \colorbox[rgb]{0.73,0.75,0.95}{\strut naive} \colorbox[rgb]{0.41,0.47,0.95}{\strut \textcolor{white}{I}} \colorbox[rgb]{0.72,0.75,0.95}{\strut am.} \colorbox[rgb]{0.76,0.78,0.95}{\strut All} \colorbox[rgb]{0.00,0.09,0.95}{\strut \textcolor{white}{I}} \colorbox[rgb]{0.55,0.59,0.95}{\strut \textcolor{white}{'ve}} \colorbox[rgb]{0.59,0.63,0.95}{\strut realized} \colorbox[rgb]{0.79,0.81,0.95}{\strut is} \colorbox[rgb]{0.84,0.85,0.95}{\strut that} \colorbox[rgb]{0.77,0.79,0.95}{\strut you} \colorbox[rgb]{0.84,0.85,0.95}{\strut are} \colorbox[rgb]{0.84,0.85,0.95}{\strut n't} actually \colorbox[rgb]{0.84,0.85,0.95}{\strut prepared} to have an \colorbox[rgb]{0.80,0.81,0.95}{\strut actual} \colorbox[rgb]{0.68,0.71,0.95}{\strut discussion} \colorbox[rgb]{0.78,0.79,0.95}{\strut .} 

\noindent\colorbox[rgb]{0.69,0.71,0.95}{\strut (OOV\_comment\_begin)} \colorbox[rgb]{0.74,0.76,0.95}{\strut What} \colorbox[rgb]{0.52,0.56,0.95}{\strut \textcolor{white}{god}} \colorbox[rgb]{0.76,0.78,0.95}{\strut do} \colorbox[rgb]{0.76,0.78,0.95}{\strut you} \colorbox[rgb]{0.65,0.68,0.95}{\strut believe} \colorbox[rgb]{0.77,0.79,0.95}{\strut in} \colorbox[rgb]{0.65,0.68,0.95}{\strut ?} \colorbox[rgb]{0.65,0.68,0.95}{\strut And} \colorbox[rgb]{0.15,0.23,0.95}{\strut \textcolor{white}{it}} \colorbox[rgb]{0.76,0.78,0.95}{\strut 's} \colorbox[rgb]{0.75,0.77,0.95}{\strut not} \colorbox[rgb]{0.71,0.73,0.95}{\strut a} \colorbox[rgb]{0.56,0.60,0.95}{\strut \textcolor{white}{fallacy}} \colorbox[rgb]{0.71,0.73,0.95}{\strut when} \colorbox[rgb]{0.56,0.60,0.95}{\strut \textcolor{white}{it}} \colorbox[rgb]{0.83,0.84,0.95}{\strut 's} \colorbox[rgb]{0.81,0.82,0.95}{\strut very} comparable to the most popular \colorbox[rgb]{0.83,0.84,0.95}{\strut gods} . 

\noindent\colorbox[rgb]{0.77,0.78,0.95}{\strut (OOV\_comment\_begin)} \colorbox[rgb]{0.67,0.70,0.95}{\strut You} \colorbox[rgb]{0.79,0.80,0.95}{\strut 're} \colorbox[rgb]{0.79,0.80,0.95}{\strut making} \colorbox[rgb]{0.81,0.83,0.95}{\strut an} \colorbox[rgb]{0.71,0.73,0.95}{\strut assumption} \colorbox[rgb]{0.78,0.79,0.95}{\strut on} \colorbox[rgb]{0.56,0.60,0.95}{\strut \textcolor{white}{what}} \colorbox[rgb]{0.61,0.65,0.95}{\strut I} \colorbox[rgb]{0.80,0.81,0.95}{\strut believe} , then \colorbox[rgb]{0.54,0.58,0.95}{\strut \textcolor{white}{attacking}} \colorbox[rgb]{0.68,0.71,0.95}{\strut your} \colorbox[rgb]{0.68,0.71,0.95}{\strut assumption} \colorbox[rgb]{0.80,0.81,0.95}{\strut of} \colorbox[rgb]{0.62,0.65,0.95}{\strut what} \colorbox[rgb]{0.74,0.76,0.95}{\strut my} \colorbox[rgb]{0.72,0.74,0.95}{\strut belief} \colorbox[rgb]{0.80,0.82,0.95}{\strut is} \colorbox[rgb]{0.80,0.82,0.95}{\strut without} \colorbox[rgb]{0.78,0.80,0.95}{\strut me} \colorbox[rgb]{0.81,0.82,0.95}{\strut even} \colorbox[rgb]{0.61,0.64,0.95}{\strut telling} \colorbox[rgb]{0.68,0.71,0.95}{\strut you} \colorbox[rgb]{0.60,0.64,0.95}{\strut anything.} \colorbox[rgb]{0.67,0.70,0.95}{\strut OOV} \colorbox[rgb]{0.51,0.55,0.95}{\strut \textcolor{white}{It}} \colorbox[rgb]{0.70,0.72,0.95}{\strut is} \colorbox[rgb]{0.75,0.77,0.95}{\strut a} \colorbox[rgb]{0.72,0.74,0.95}{\strut OOV} \colorbox[rgb]{0.53,0.57,0.95}{\strut \textcolor{white}{It}} \colorbox[rgb]{0.73,0.75,0.95}{\strut 's} \colorbox[rgb]{0.76,0.78,0.95}{\strut the} \colorbox[rgb]{0.38,0.44,0.95}{\strut \textcolor{white}{comparison}} \colorbox[rgb]{0.59,0.63,0.95}{\strut itself} that is \colorbox[rgb]{0.80,0.82,0.95}{\strut OOV} \colorbox[rgb]{0.83,0.84,0.95}{\strut If} they were n't \colorbox[rgb]{0.84,0.85,0.95}{\strut comparable} at all \colorbox[rgb]{0.85,0.86,0.95}{\strut ,} \colorbox[rgb]{0.82,0.84,0.95}{\strut then} \colorbox[rgb]{0.73,0.76,0.95}{\strut it} 'd be impossible to commit the \colorbox[rgb]{0.85,0.86,0.95}{\strut OOV} \colorbox[rgb]{0.73,0.76,0.95}{\strut You} can compare apples to oranges \colorbox[rgb]{0.80,0.82,0.95}{\strut ,} \colorbox[rgb]{0.51,0.56,0.95}{\strut \textcolor{white}{but}} the \colorbox[rgb]{0.84,0.85,0.95}{\strut moment} you use your fingernails to peel an apple you \colorbox[rgb]{0.69,0.71,0.95}{\strut look} like an \colorbox[rgb]{0.72,0.74,0.95}{\strut idiot} . 
\end{small}
\caption{\label{fig:prediction.heat.map} An example of reconstructed word weight heat map extracted from the attention matrix for a thread which ends up in ad hominem; three previous arguments are shown (see Figure \ref{fig:thread-sampling} for sampling details).}
\end{figure}



In the following analysis, we solely relied on the weights of words or phrases learned by the attention model, see an example in Figure \ref{fig:prediction.heat.map}. Based on our observations, we summarize several linguistic and argumentative phenomena with examples most likely responsible for ad hominem threads in Table \ref{tab:triggers.of.ah}.

The identified phenomena have few interesting properties in common. First, they all are topic-independent rhetorical devices (except for the loaded keywords at the bottom). Second, many of them deal with meta-level argumentation, i.e., arguing about argumentation (such as missing support or fallacy accusations). Third, most of them do not contain profanity (in contrast to the actual ad hominem arguments of which a third are vulgar insults; cf.\ Table \ref{tab:typology-spans}). And finally, all of them should be easy to avoid.





\begin{table*}
\centering
\begin{small}
\begin{tabular}{p{0.24\textwidth}p{0.70\textwidth}}
\toprule
\textbf{Phenomena} & \textbf{Examples} \\ \midrule
Introducing vulgar intensifiers or interrogatives &
\textbf{388(-1)} \emph{``Where the fuck is your idea to ...''},
\textbf{712(-2)} \emph{``no shortage of fucking gun''},
\textbf{1277(-1)} \emph{``This is fucking CMV''},
\textbf{428(-2)} \emph{``I'm fucking trans!''},
\textbf{2018(-2)} \emph{``an arrogant fuck''},
\textbf{1277(-2)} \emph{``What the fuck are you smoking?''} \\ \midrule
Direct imperatives & 
\textbf{1003(-3)} \emph{``You should get more mad about it"}, 
\textbf{857(-2)} \emph{``You need to do a lot better than that."}, 
\textbf{233(-2)} \emph{``So now delete your post"}, 
\textbf{749(-1)} \emph{``google this fact as well"}, 
\textbf{1276(-1)} \emph{``Just look back at the reasons why ..."} \\ \midrule
Accusing of believing in or using propaganda &
\textbf{522(-1)} \emph{``It's right wing propaganda?"},
\textbf{1003(-1)} \emph{``If you're not outraged, you're not paying attention to our propaganda that says the opposite of literally thousands of published research articles"} \\ \midrule
Accusation of fallacies or bad argumentation practice &
\textbf{238(-3)} \emph{``your snide remarks and poor argumentation skills"},
\textbf{1117(-2)} \emph{``you're circle jerking A vs. B"},
\textbf{263(-3)} \emph{``You're grasping at straws"},
\textbf{78(-3)} \emph{``You sure like changing words and statements to make your argument appear more cogent, don't you?"},
\textbf{210(-1)} \emph{``Your arguments range from ... to ..."},
\textbf{1085(-3)} \emph{``It's only a fallacy"},
\textbf{144(-1)} \emph{``You haven't presented any evidence or argument that disagrees with anything I've said."},
\textbf{587(-3)} \emph{``If only you wouldn't rely on fallacious arguments"} \\ \hline
Reinterpreting opponent's positions &
\textbf{982(-1)} \emph{``The fact that you obviously think ... reveals ..."},
\textbf{982(-2)} \emph{``What makes you think I see myself ... ?"},
\textbf{1060(-3)} \emph{``That kind of thinking is ..."},
\textbf{760(-1)} \emph{``If I'm understanding you correctly"},
\textbf{405(-1)} \emph{``... deluded yourself into believing factually incorrect things"}
\textbf{587(-1)} \emph{``You're making an assumption on what I believe, then attacking your assumption of what my belief is without me even telling you anything."} \\ \hline
Accusation of not reading the other party's arguments &
\textbf{586(-1)} \emph{``... me without even reading my ..."},
\textbf{240(-1)} \emph{``You are just reading it wrong."},
\textbf{310(-1)} \emph{``Oh, you're not actually reading my ..."} \\ \midrule
Pointing at missing or unsupported evidence and facts &
\textbf{1238(-2)} \emph{``unsupported bullshit as before"},
\textbf{1121(-3)} \emph{``you can't chose your facts"},
\textbf{931(-1)} \emph{``If that's your only argument ..."},
\textbf{486(-2)} \emph{``unsubstantiated statement"},
\textbf{486(-1)} \emph{``unsupported claims"},
\textbf{71(-2)} \emph{``factually correct"},
\textbf{915(-1)} \emph{``But for the sake of argument, your points are pitifully .."},
\textbf{388(-3)} \emph{``Please provide statistics ... It's silly to debate statistics without actual numbers."} \\ \midrule
UPPERCASE &
\textbf{1238(-3)} \emph{``NO ONE CLAIMED THAT ... ARE NOT ... AGAINST ..."} \\ \midrule
Sarcasm &
\textbf{78(-2)} \emph{``But I'm sure you know best"},
\textbf{310(-1)} \emph{``Have a nice day."},
\textbf{1276(-1)} \emph{``Good luck with that"} \\ \midrule
Mentions of trolling & 
\textbf{701(-2)} \emph{``Then you are giving trolls the victory then?"} \\ \midrule
Loaded keywords &
\emph{Nazi, rape, racist} \\ \bottomrule
\end{tabular}
\end{small}
\caption{\label{tab:triggers.of.ah} Phenomena resulting into ad hominem learned by the SSAE-NN model. The first number is the instance ID (available in the supplementary material), the minus number in parentheses is the position of the argument before the ad hominem.}
\end{table*}




\paragraph{Misleading `features'}

False positives revealed properties that misled the network to classify delta threads as ad hominem threads.

\begin{itemize}

\item These include \textbf{topic words} (such as \emph{racism, blacks, slave, abortion}) which reflects the implicit bias in the data.
\item Actual interest mixed with indifference in \textbf{sarcasm} is also problematic (\textbf{185(-2)} \emph{``That's a very interesting ..."}).
\item Another problematic phenomena is also \textbf{expressed disagreement} (\textbf{678(-2)} \emph{``overheated rhetoric"}, \textbf{203(-2)} \emph{``But I suppose this argument is ..."}, \textbf{230(-2)} \emph{``But I don't think it's quite ..."}, \textbf{938(-1)} \emph{``I disagree too, however~..."}).
\end{itemize}

False negatives were caused basically by presence of many `informative' \textbf{content words} (\textbf{980} \emph{unemployment, quarterly publication, inflation data}, \textbf{474} \emph{actual publications, this experiment, biological ailments, medical doctorate}, \textbf{1214}
\emph{graduate degree, education, health insurance}) and \textbf{misinterpreted sarcasm} (\textbf{285(-1)}
\emph{``Also this is a cute analogy''}).


\section{Conclusion}


In this article, we investigated ad hominem argumentation on three levels of discourse complexity. We looked into qualitative and quantative properties of ad hominem arguments, crowdsourced labeled data, experimented with models for prediction (0.810 accuracy; \ref{sec:recognizing.ah}), and proposed an updated typology of ad hominem properties (\ref{sec:typology.of.ah}). We then looked into the dynamics of argumentation to examine the relation between the quality of the original post and immediate ad hominem arguments (\ref{sec:triggers.first.level}). Finally, we exploited the learned representation of Self-Attentive Embedding Neural Network to search for features triggering ad hominem in one-to-one discussions. We found several categories of rhetorical devices as well as misleading features (\ref{sec:before.calling.names:results}).

There are several points that deserve further investigation. First, we have ignored meta-information of the debate participants, such as their overall activity (i.e., whether they are spammers or trolls). Second, the proposed typology of ad hominem causes has not yet been post-verified empirically. Third, we expect that personality traits of the participants (BIG5) may also play a significant role in the argumentative exchange. We leave these points for future work.

We believe that our findings will help gain better understanding of, and hopefully keep restraining from, ad hominem fallacies in good-faith discussions.



\section*{Acknowledgments}

This work has been supported by the ArguAna Project GU~798/20-1 (DFG), and by the DFG-funded research training group ``Adaptive Preparation of Information form Heterogeneous Sources'' (AIPHES, GRK 1994/1).



\appendix

\chapter{Some cool appendix}

\lipsum[1]

\backmatter


\printbibliography


\end{document}